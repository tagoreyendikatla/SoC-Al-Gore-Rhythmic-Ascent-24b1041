\documentclass[12pt, letterpaper]{article}
\usepackage[utf8]{inputenc}

\title{Report of WEEK 3}
\author{
    Y Tagore Ravindra Sree 
}
\date{} % Add date if needed

\begin{document}
\maketitle
\section{Searching Algorithms}
\subsection{Linear Search}
\begin{itemize}
    \item Checks each element one by one.
    \item Works on unsorted lists.
    \item Time Complexity: O(n) (worst and average case).
\end{itemize}
\subsection{Binary Search}
\begin{itemize}
    \item Works on sorted arrays/lists.
    \item Uses a divide and conquer approach.
    \item Time Complexity: O(log n) (worst and average case)
\end{itemize}
\subsection{Jump Search}
\begin{itemize}
    \item Works on sorted arrays.
    \item Jumps in fixed steps, then performs linear search. Step size is square root of the target.
    \item Time Complexity: O(sqrt(n))
\end{itemize}
\subsection{Exponential Search}
\begin{itemize}
    \item Works on sorted arrays.
    \item Doubles the search range, then performs binary search.
    \item Time Complexity: O(log n)
\end{itemize}
\subsection{Ternary Search}
\begin{itemize}
    \item Similar to binary search but divides the array into three parts.
    \item Works on sorted arrays.
    \item Time Complexity: O(log n)
\end{itemize}
\subsection{Interpolation Search}
\begin{itemize}
    \item Works on sorted, uniformly distributed data.
    \item Esitmates the postion of the key(better than binary search for some cases).
    \item Time Complexity: O(log log n)
\end{itemize}
\subsection{DFS}
\begin{itemize}
    \item Used for tree/graph traversal.
    \item Explores as far as possible before backtracking.
    \item Time Complexity:O(V+E)(vertices+edges).
\end{itemize}
\subsection{BFS}
\begin{itemize}
    \item Also for tree/graph traversal.
    \item Explores all neighbors at the current level before moving deeper.
    \item Time Complexity:O(V+E)
\end{itemize}
\end{document}